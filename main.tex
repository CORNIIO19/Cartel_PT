\documentclass[final]{beamer}

\usepackage[size=custom,width=90,height=119.38,scale=1.6]{beamerposter}
\usetheme{ConfPoster}

\usepackage[spanish]{babel}
\usepackage{graphicx}
\usepackage{booktabs}
\usepackage{multicol}
\usepackage{lipsum}
\usepackage{wrapfig}
\usepackage{tcolorbox}
\usepackage{enumitem}

\setlength{\columnsep}{1em}
\setlength{\columnseprule}{0mm}

\title{Gestor inteligente para notas academicas}
\author{Omar Jhonatan Sosa Bobadilla}
\institute{Universidad Autónoma Metropolitana, Unidad Cuajimalpa}
\begin{document}
\begin{frame}

\begin{columns}[t]

% ============================================================
% COL 1
% ============================================================

\begin{column}{0.40\textwidth}

% ------------------------------------------------------------
\begin{block}{\Large Introducción}
El usuario acumula una gran cantidad de notas provenientes de:
\begin{itemize}
    \item Materias escolares
    \item Tareas y proyectos
    \item Notas personales
    \item Colaboraciones con otras personas
\end{itemize}

Esto provoca dificultades al momento de consultarlas o relacionarlas.
\end{block}

% ------------------------------------------------------------
\begin{block}{\Large Justificación}
\begin{itemize}[leftmargin=*]
    \item \textbf{Transporte}: las notas se encuentran en diferentes dispositivos o formatos.
    \item \textbf{Organización}: difícil clasificación manual.
    \item \textbf{Estudio}: la información no se encuentra cuando se necesita.
\end{itemize}

Estas situaciones generan frustración y pérdida de tiempo.
\end{block}

% ------------------------------------------------------------

\begin{block}{\Large Estado del Arte}
Desarrollar un sistema inteligente capaz de:
\begin{itemize}[leftmargin=*]
    \item Centralizar y almacenar notas.
    \item Clasificarlas automáticamente mediante IA.
    \item Permitir consultas rápidas, precisas y personalizadas.
\end{itemize}
\end{block}

\begin{center}
\includegraphics[width=0.8\linewidth]{images/Contexto_2.png}
\end{center}
\end{column}




% ============================================================
% COL 2
% ============================================================

\begin{column}{0.40\textwidth}

% ------------------------------------------------------------
\begin{block}{\Large Metodología}

El proceso general del usuario incluye:

\begin{itemize}[leftmargin=*]
    \item Creación constante de notas personales, académicas y colaborativas.
    \item Necesidad de consultar notas rápidamente.
    \item Problema de fragmentación de información.
    \item Inicio del uso de la aplicación para centralizar y organizar contenido.
\end{itemize}

La aplicación permite transformar este proceso caótico en uno estructurado y accesible.

El sistema utiliza técnicas de procesamiento de lenguaje natural para:

\begin{itemize}[leftmargin=*]
    \item Analizar similitudes semánticas entre palabras clave.
    \item Detectar temas relevantes en cada nota.
    \item Asignar etiquetas o categorías automáticas.
\end{itemize}

Las notas terminan agrupadas dentro de una base de conocimiento estructurada.
\end{block}

\begin{block}{\Large Uso de la Aplicación}
El usuario puede realizar consultas como:

\begin{quote}
\emph{``¿Qué vimos en Sistemas Operativos el 31-10-25?''}
\end{quote}

La IA analiza la base de conocimiento y devuelve:

\begin{itemize}[leftmargin=*]
    \item \textbf{Respuesta positiva}: lista de temas relacionados.
    \item \textbf{Respuesta negativa}: indica que no se encontraron notas para esa fecha o categoría.
\end{itemize}

\end{block}

% ------------------------------------------------------------
\begin{block}{\Large Propuesta}
La aplicación resuelve problemas reales de organización y consulta de información mediante:
\begin{itemize}[leftmargin=*]
    \item Centralización de notas.
    \item Clasificación automática con IA.
    \item Consultas naturales y personalizadas.
\end{itemize}

Es una herramienta útil para estudiantes y profesionales que manejan grandes volúmenes de información.
\end{block}

\end{column}


\end{columns}

\end{frame}
\end{document}
