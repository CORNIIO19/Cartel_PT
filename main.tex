\documentclass[final]{beamer}

\usepackage[size=custom,width=90,height=119.38,scale=1.6]{beamerposter}
\usetheme{ConfPoster}

\usepackage[spanish]{babel}
\usepackage[utf8]{inputenc}
\usepackage{graphicx}
\usepackage{booktabs}
\usepackage{multicol}
\usepackage{lipsum}
\usepackage{wrapfig}
\usepackage{tcolorbox}
\usepackage{enumitem}

% Espacio horizontal entre columnas (ajústalo a tu gusto)
\setlength{\columnsep}{1cm} % antes: 1em
\setlength{\columnseprule}{0mm}

\title{Gestor inteligente para notas academicas}
\author{Omar Jhonatan Sosa Bobadilla}
\institute{Universidad Autónoma Metropolitana, Unidad Cuajimalpa}
\begin{document}
\begin{frame}


\begin{columns}[t]

% ============================================================
% COL 1
% ============================================================

\begin{column}{0.44\textwidth}

% ------------------------------------------------------------
    
\begin{block}{\Large Introducción}
    En el ámbito académico, los estudiantes generan y consultan apuntes para aprender, pero cuando el volumen de información crece, los medios físicos dificultan su organización, acceso y consulta. Las herramientas digitales modernas, como Notion u Obsidian, mejoran esta experiencia, aunque presentan limitaciones: dependen de internet, no permiten consultas personalizadas y no garantizan seguridad ni control total sobre los datos, lo que pone en riesgo la información almacenada.

    \vspace{.5cm}

    Además, muchos estudiantes enfrentan dificultades para organizar sus conocimientos y desconocen métodos efectivos para estudiar o generar apuntes de calidad, lo que genera incertidumbre sobre su aprendizaje.

    \vspace{.5cm}

    Ante estas problemáticas, se plantea desarrollar una aplicación inteligente de gestión de notas que funcione offline, garantice privacidad y seguridad, permita crear una base de conocimiento personalizada y ofrezca recomendaciones de aprendizaje apoyadas en tecnologías como procesamiento local del lenguaje natural y sistemas distribuidos. Esta herramienta busca ofrecer una experiencia confiable, accesible y centrada en el usuario.
\end{block}

% ------------------------------------------------------------
\begin{block}{\Large Justificación}

    La aplicación permitirá crear, organizar y consultar una base de conocimiento personalizada con total privacidad, incorporando tecnologías como bases de datos vectoriales para búsquedas semánticas, modelos de lenguaje locales y sistemas distribuidos, además de ofrecer funciones offline para mejorar la accesibilidad.

    \vspace{.5cm}

    El Gestor inteligente para notas académicas responde a la necesidad de estructurar y consultar información de forma segura, personalizada y eficiente, siendo útil para estudiantes, docentes y profesionales.

    Su relevancia se sustenta en varios puntos clave:
    
    \begin{itemize}[leftmargin=*]
        \item Privacidad y control de datos: el procesamiento local reduce riesgos de filtraciones y mejora el rendimiento.
        \item Accesibilidad offline: permite consultar información desde cualquier lugar sin depender de internet.
        \item Colaboración descentralizada: facilita el intercambio de conocimiento sin una arquitectura centralizada.
        \item Procesamiento de lenguaje natural: potencia funciones avanzadas de búsqueda y análisis de contenido.
        \item Mejor experiencia de usuario: soporta información multimodal para una interacción más flexible.
        \item Disponibilidad multiplataforma: accesible en distintos dispositivos y sistemas operativos.   
    \end{itemize}

\end{block}


\begin{block}{\Large Estado del Arte}
Desarrollar un sistema inteligente capaz de:
\begin{itemize}[leftmargin=*]
    \item Centralizar y almacenar notas.
    \item Clasificarlas automáticamente mediante IA.
    \item Permitir consultas rápidas, precisas y personalizadas.
\end{itemize}

%tabla comparativa
\begin{table}
\centering
\begin{tabular}{l|ccc}
\toprule
\textbf{Características} & \textbf{Evernote} & \textbf{Notion} & \textbf{Propuesta} \\
\midrule
Almacenamiento centralizado & Sí & Sí & Sí \\
Clasificación automática & No & No & Sí \\
Consultas personalizadas & Limitado & Limitado & Sí \\
Interfaz amigable & Sí & Sí & Sí \\
Soporte multiplataforma & Sí & Sí & Sí \\
\bottomrule
\end{tabular}

\end{table}

\end{block}


\end{column}


%col 2 reales ----------------------------------------------------------




% ============================================================
% COL 2
% ============================================================

\begin{column}{0.44\textwidth}

% ------------------------------------------------------------
\begin{block}{\Large Metodología}

El sistema utiliza técnicas de procesamiento de lenguaje natural para:

\begin{itemize}[leftmargin=*]
    \item Analizar similitudes semánticas entre palabras clave.
    \item Detectar temas relevantes en cada nota.
    \item Asignar etiquetas o categorías automáticas.
\end{itemize}

Las notas terminan agrupadas dentro de una base de conocimiento estructurada.
\end{block}



% ------------------------------------------------------------
\begin{block}{\Large Propuesta}
El usuario puede realizar consultas como:

\begin{quote}
\emph{\textquestiondown Qué vimos en Sistemas Operativos el 31-10-25?}
\end{quote}

La IA analiza la base de conocimiento y devuelve:

\begin{itemize}[leftmargin=*]
    \item \textbf{Respuesta positiva}: lista de temas relacionados.
    \item \textbf{Respuesta negativa}: indica que no se encontraron notas para esa fecha o categoría.
\end{itemize}
La aplicación resuelve problemas reales de organización y consulta de información mediante:
\begin{itemize}[leftmargin=*]
    \item Centralización de notas.
    \item Clasificación automática con IA.
    \item Consultas naturales y personalizadas.
\end{itemize}

Es una herramienta útil para estudiantes y profesionales que manejan grandes volúmenes de información.
\end{block}

\begin{center}
\includegraphics[width=0.8\linewidth]{images/Contexto_2.png}
\end{center}

\end{column}


\end{columns}

\end{frame}
\end{document}
