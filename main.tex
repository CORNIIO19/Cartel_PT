\documentclass[final]{beamer}

\usepackage[size=custom,width=90,height=119.38,scale=1.6]{beamerposter}
\usetheme{ConfPoster}

\usepackage[spanish]{babel}
\usepackage[utf8]{inputenc}
\usepackage{graphicx}
\usepackage{booktabs}
\usepackage{multicol}
\usepackage{lipsum}
\usepackage{wrapfig}
\usepackage{tcolorbox}
\usepackage{enumitem}

\colorlet{titlefg}{DarkGreen}
\setbeamersize{text margin left=3cm,text margin right=3cm}
\setlength{\columnsep}{1cm}

\title{Gestor inteligente para notas academicas}
\author{Omar Jhonatan Sosa Bobadilla}
\institute{Universidad Autónoma Metropolitana, Unidad Cuajimalpa}

\begin{document}

\begin{frame}[t]

    \begin{block}{\Large Introducción}
            \vspace{1cm}
            
            Los estudiantes utilizan desde papel hasta herramientas digitales como Notion, Obsidian u OneNote para gestionar información, pero estas plataformas dependen de internet y no ofrecen funciones avanzadas como consultas personalizadas o contextos privados. Además, al usar servicios externos, los usuarios pierden control sobre la privacidad y seguridad de sus datos.\\
            \end{block}

\begin{minipage}[t]{\textwidth}

    \begin{minipage}[t]{0.47\textwidth}

        \begin{block}{\Large Metodología}
\vspace{.5cm}
El sistema utiliza técnicas de procesamiento de lenguaje natural para:

\begin{itemize}[leftmargin=*]
    \item Analizar similitudes semánticas entre palabras clave.
    \item Detectar temas relevantes en cada nota.
    \item Asignar etiquetas o categorías automáticas.
\end{itemize}

Las notas terminan agrupadas dentro de una base de conocimiento estructurada.
\end{block}
        

\end{minipage}\hspace{0.02\textwidth}
% ------------------------------------------------------------
\begin{minipage}[t]{0.47\textwidth}
\begin{block}{\Large Justificación}
    \vspace{1cm}

    %La aplicación permitirá crear, organizar y consultar una base de conocimiento personalizada con total privacidad, incorporando tecnologías como bases de datos vectoriales para búsquedas semánticas, modelos de lenguaje locales y sistemas distribuidos, además de ofrecer funciones offline para mejorar la accesibilidad.

    El \emph{Gestor inteligente para notas académicas} responde a la necesidad de estructurar y consultar información de forma segura, personalizada y eficiente, siendo útil para estudiantes, docentes y profesionales.
    
    \vspace{.5cm}
    \begin{itemize}[leftmargin=*]
        \item Privacidad y control de datos.
        \item Accesibilidad offline.
        \item Colaboración descentralizada.
        \item Procesamiento de lenguaje natural.
        \item Mejor experiencia de usuario.
        \item Disponibilidad multiplataforma.   
    \end{itemize}

\end{block}
\end{minipage}

\end{minipage}

% ============================================================
% COL 2
% ============================================================

\begin{minipage}[t][0.3\textheight][t]{\textwidth}
    \vspace{1cm}


\begin{block}{\Large Estado del Arte}
    \vspace{1cm}
Las principales caracteristicas que diferencian el \emph{Gestor inteligente para notas académicas} con respecto a las soluciones encontradas en el mercado.
\vspace{1cm}
%tabla comparativa
\begin{table}[H]
\centering
\small
\setlength{\tabcolsep}{50pt}
\renewcommand{\arraystretch}{1.20}
\begin{tabular}{|l|c|c|c|c|}
\toprule
\textbf{Características} & \textbf{Notion} & \textbf{Obsidian} & \textbf{NotebookLM} & \textbf{Propuesta} \\
\midrule
Enfoque principal & Productividad & Conocimiento personal & Resúmenes con IA & Gestión académica con IA local \\
\hline
Privacidad & Baja (nube) & Media (local + sync) & Baja (Google) & Alta (100\% local) \\
\hline
Acceso sin conexión & Limitado & Parcial & No & Total \\
\hline
IA integrada & Sí (nube) & No nativa & Sí (nube) & Sí (local) \\
\hline
Colaboración & Alta & Limitada & No & Distribuida \\
\hline
Facilidad de uso & Intuitiva & Técnica & Simple & Ligera e intuitiva \\
\hline
Enfoque educativo & General & Investigación personal & Lectura/síntesis & Estudiantes y docentes \\
\hline
Soporte multiplataforma & Sí & Sí & Sí & Sí \\
\bottomrule
\end{tabular}
\end{table}

\vspace{.5cm}

\end{block}

% ------------------------------------------------------------
\begin{minipage}[t]{0.48\textwidth}


% ------------------------------------------------------------
\begin{block}{\Large Propuesta}
\vspace{.5cm}
    
Por ello, se plantea una aplicación inteligente que funcione offline, permita crear una base de conocimiento personalizada y asegure la protección de la información mediante tecnologías como procesamiento local del lenguaje, interfaces especializadas y sistemas distribuidos. El usuario puede realizar consultas como:

\begin{quote}
\emph{\textquestiondown~Qué vimos en Sistemas Operativos el 31--10--25?}
\end{quote}

La IA analiza la base de conocimiento y devuelve:

\begin{itemize}[leftmargin=*]
    \item \textbf{Respuesta positiva}: lista de temas relacionados.
    \item \textbf{Respuesta negativa}: indica que no se encontraron notas para esa fecha o categoría.
\end{itemize}
La aplicación resuelve problemas reales de organización y consulta de información mediante:
\begin{itemize}[leftmargin=*]
    \item Centralización de notas.
    \item Clasificación automática con IA.\@
    \item Consultas naturales y personalizadas.
\end{itemize}

\end{block}
\end{minipage}\hfill
\begin{minipage}[t]{0.48\textwidth}

\begin{block}{\large Contexto de uso}
    \begin{center}
    \includegraphics[width=.68\linewidth]{images/Contexto_2.png}
    \end{center}
    \end{block}
\end{minipage}

\end{minipage}

\end{frame}
\end{document}
