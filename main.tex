\documentclass[final]{beamer}

\usepackage[size=custom,width=90,height=119.38,scale=1.6]{beamerposter}
\usetheme{ConfPoster}

\usepackage[spanish]{babel}
\usepackage[utf8]{inputenc}
\usepackage{graphicx}
\usepackage{booktabs}
\usepackage{multicol}
\usepackage{lipsum}
\usepackage{wrapfig}
\usepackage{tcolorbox}
\usepackage{enumitem}

\colorlet{titlefg}{DarkGreen}
\setbeamersize{text margin left=3cm,text margin right=3cm}
\setlength{\columnsep}{1cm}

\title{Gestor inteligente para notas academicas}
\author{Omar Jhonatan Sosa Bobadilla}
\institute{Universidad Autónoma Metropolitana, Unidad Cuajimalpa}

\begin{document}

\begin{frame}[t]
    \vspace{.5cm}
    \begin{block}{\Large 1. Introducción}
    Las herramientas digitales actuales para tomar notas no garantizan privacidad, autonomía ni acceso sin conexión. Esto es relevante porque los estudiantes dependen cada vez más de estos sistemas para gestionar grandes volúmenes de información. Sin embargo, las plataformas existentes requieren internet, no permiten consultas personalizadas y exponen los datos a riesgos si el servicio cambia o desaparece. Por ello se necesita una solución local, privada y capaz de realizar consultas inteligentes.
            \end{block}

% \vspace{0.5cm}
\begin{tcolorbox}[colframe=RoyalBlue,colback=RoyalBlue,height=0.25cm,left=0pt,right=0pt,top=0pt,bottom=0pt,boxsep=0pt]
\end{tcolorbox}

\begin{minipage}[t]{\textwidth}

\begin{minipage}[t]{0.47\textwidth}

\begin{block}{\Large 2. Metodología}

El proyecto se desarrolla con metodologías ágiles (AUP y Scrum), permitiendo iteraciones y retroalimentación continua.
Las actividades principales incluyen:
\begin{itemize}[leftmargin=*]
    \item Definición de requisitos y diseño de arquitectura.
    \item Prototipado y pruebas
    \item Construcción incremental en sprints
\end{itemize}

Para comprender necesidades reales de estudiantes, se aplican encuestas, entrevistas y análisis documental

\end{block}

\vspace{.5cm}
   \begin{block}{\Large 3. Justificación}

   Las herramientas actuales dependen de la nube, lo que limita la privacidad, el acceso sin internet y el control de los estudiantes sobre sus notas. Por ello, se requiere una alternativa local y segura que permita gestionar y consultar información académica sin depender de servicios externos
\end{block}

\vspace{1cm}

\begin{block}{\Large 4. Propuesta}
    La aplicación centraliza todas las notas en una base de conocimiento local y permite consultarlas mediante un chat con IA sin conexión a internet. Sus funciones principales incluyen:
    
    \begin{itemize}[leftmargin=*]

    \item Acceso offline

    \item Clasificación automática

    \item Consultas en lenguaje natural

    \item Colaboración distribuida

    \item Privacidad y control total del usuario
    \end{itemize}

Esta solución aborda problemas reales como desorden, falta de contexto y difícil acceso a la información.

% La IA analiza la base de conocimiento y devuelve:

% \begin{itemize}[leftmargin=*]
%     \item \textbf{Respuesta positiva}: lista de temas relacionados.
%     \item \textbf{Respuesta negativa}: indica que no se encontraron notas para esa fecha o categoría.
% \end{itemize}
% La aplicación resuelve problemas reales de organización y consulta de información mediante:
% \begin{itemize}[leftmargin=*]
%     \item Centralización de notas.
%     \item Clasificación automática con IA.\@
%     \item Consultas naturales y personalizadas.
% \end{itemize}

\end{block}
        

\end{minipage}
\hspace{0.02\textwidth}%
{\textcolor{RoyalBlue}{\vrule width 0.25cm height 1.5cm}}%
\hspace{0.02\textwidth}%
% ------------------------------------------------------------
\begin{minipage}[t]{0.48\textwidth}
    \begin{block}{\large 5. Funcionamiento general}
    \begin{center}
    \includegraphics[width=.89\linewidth]{images/Contexto_2.png}
    \end{center}
    \end{block}

\end{minipage}

\end{minipage}

% ============================================================
% COL 2
% ============================================================

\begin{minipage}[t][0.3\textheight][t]{\textwidth}
% % ------------------------------------------------------------
% \begin{minipage}[t]{0.48\textwidth}

% \end{minipage}\hfill
\begin{minipage}[t]{0.48\textwidth}

\end{minipage}

% \vspace{0.5cm}
\noindent\begin{tcolorbox}[colframe=RoyalBlue,colback=RoyalBlue,height=0.3cm,left=0pt,right=0pt,top=0pt,bottom=0pt,boxsep=0pt]
\end{tcolorbox}
% \vspace{0.5cm}

\begin{minipage}[t]{0.48\textwidth}
\begin{block}{\Large 6. Estado del Arte}

Las caracteristicas que diferencian al \emph{Gestor inteligente para notas académicas} con soluciones encontradas en el mercado son:
%tabla comparativa
\begin{table}[H]
\centering
\footnotesize
\setlength{\tabcolsep}{25pt} % Espaciado entre columnas
\renewcommand{\arraystretch}{1.40} % Altura de filas
\begin{tabular}{|l|c|c|c|c|}
\toprule
\textbf{Características} & \textbf{Notion} & \textbf{Obsidian} & \textbf{NotebookLM} & \textbf{Propuesta} \\
\midrule
Almacenamiento & Nube & Parcial & No & \textbf{Sí (100\%)} \\
\hline
Privacidad & No & Parcial & No & \textbf{Sí} \\
\hline
Acceso sin conexión & Limitado & Sí & No & \textbf{Sí} \\
\hline
IA integrada & Nube & Plugins & Nube & \textbf{Sí (local)} \\
\hline
Consultas  & Sí & No nativo & Sí & \textbf{Sí (local)} \\
\hline
Colaboración & Sí (centralizada) & Limitada & No & \textbf{Distribuida} \\
\hline
Enfoque educativo & No & No & No & \textbf{Sí} \\
\bottomrule
\end{tabular}
\end{table}

\vspace{.5cm}

\end{block}

\end{minipage}\hspace{0.04\textwidth}%
{\textcolor{RoyalBlue}{\vrule width 0.25cm height 0cm}}%
\hspace{0.04\textwidth}%
\begin{minipage}[t]{0.46\textwidth}
\begin{block}{\Large 7. Resultados Esperados}

    \begin{itemize}[leftmargin=*]

    \item El sistema permitirá gestionar y consultar notas de forma local, segura y sin conexión a internet.

    \item Mejorará la organización académica mediante clasificación automática y consultas en lenguaje natural.

    \item La integración de IA local aumentará la eficiencia sin comprometer la privacidad.

    \item Los estudiantes tendrán mayor control, autonomía y protección de sus datos.

    \item La solución superará las limitaciones de las herramientas actuales basadas en la nube.
    \end{itemize}
    
\end{block}


\end{minipage}


\end{minipage}

\end{frame}
\end{document}
